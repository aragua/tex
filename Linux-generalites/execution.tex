\section{Execution d'un système GNU/Linux}

\subsection{Démarrage}
\begin{frame}{Execution}{Démarrage}
	\begin{enumerate}
		\item
			Le matériel lance le bootloader
		\item
			Le bootloader configure la RAM et le support contenant le noyau 
		\item
			Le bootloader copie le noyau en RAM et lui donne la main
		\item
			Le noyau s'auto extrait en RAM
		\item
			Le noyau initialise les périphériques et les services (ordonnanceur, pile réseaux,syteme de fichier...)
		\item
			Le noyau récupère dans le rootfs le binaire d'initialisation (/sbin/init par défaut) et cré le processus 1
		\item
			Ce processus est chargé de démarrer tous les services en espaces utilisateurs.
	\end{enumerate}
\end{frame}

\subsection{Le processus init}
\begin{frame}{Execution}{Le processus init}
	Le père des processus du système\\
	Plusieurs systèmes possible:
	\begin{itemize}
		\item
			system5: 
			\begin{itemize}
				\item
					facile à configurer car basé sur des scripts shell
				\item
					architecture vieille et dépassé. Moins performant.
			\end{itemize}
		\item
			systemd:
		\begin{itemize}
			\item
				Parallélise l'initialisation du système. Centralise de nombreux services (logs,
			\item
				Plus complexe à prendre en main et moins modulaire pour certain.
		\end{itemize}
		\item
			custom:
			\begin{itemize}
				\item
					Linux permet d'utiliser son propre binaire init.
				\item
					Il faut gérer soit même le système.
			\end{itemize}	
	\end{itemize}
\end{frame}
