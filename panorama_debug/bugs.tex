%%%%%%%%%%%%%%%%%%%%%%%    BUGS    %%%%%%%%%%%%%%%%%%%%%%%
\section{Bugs et erreurs}

\subsection{Définition}

\begin{frame}{Définition}{Qu'est ce qu'un bug?}
	\begin{itemize}
		\item
			Un défaut de conception d'un programme informatique à l'origine d'un dysfonctionnement.
		\item
			Gravité : défauts d’affichage mineurs -> crash système critique (explosion du vol 501 d'Ariane 5).
		\item
			Un bug peut résider dans n'importe quel logiciel contenant du code (application, librairie, firmware, ...)
	\end{itemize}	
\end{frame}

\begin{frame}{Définition}{Classification des cas courants}
	Exemple de classification des cas courants de bugs:
	\begin{itemize}
		\item
			Erreur de segmentation
		\item
			Dépassement d'entier
		\item
			Dépassement de tampon
		\item
			Dépassement de pile
		\item
			Fuite mémoire
		\item
			Situation de compétition
		\item
			Interblocage
		\pause
		\item
			Performance dégradée
	\end{itemize}
\end{frame}

\begin{frame}{Définition}{Les erreurs}
	\begin{itemize}
		\item
			Une erreur retournée par un programme n'est pas un bug.
		\item
			C'est une fonctionnalité permettant de notifier l'utilisateur d'un comportement anormal d'un logiciel. 
		\item
			Elle peut résulter d'un mauvais paramètre d'entrée ou d'un bug logiciel.
		\item
			Plusieurs types d'erreurs rencontrés par le développeur:
			\begin{itemize}
				\item
					Erreur de compilation
				\item
					Erreur système
				\item
					Erreur de l'application (gestion des bugs)
			\end{itemize}
	\end{itemize}
\end{frame}


\subsection{Exemple}

%%%%%%  segfault  %%%%%%
\defverbatim[colored]\mycode{%
\begin{lstlisting}[style=customc]
int main (int argc, char ** argv)
{
	char ptr[10];
	memcpy(ptr, argv[1], strlen(argv[1])+1);
	printf("%s\n", ptr);
	return 0;
}
\end{lstlisting}}

\defverbatim[colored]\myshell{%
\begin{lstlisting}[style=customshell]
$ ./segfault toto
toto
$ ./segfault
Erreur de segmentation (core dumped)
$
\end{lstlisting}}

\begin{frame}[fragile]{Erreur de segmentation}
Tentative d'accès à un segment de mémoire inexistant ou non autorisé.
\mycode
\myshell
\end{frame}


%%%%%%  Dépassement d'entier  %%%%%%
\defverbatim[colored]\mycode{%
\begin{lstlisting}[style=customc]
int main (int argc, char ** argv)
{
	short s;
	for(s=0; s < atoi(argv[1]); s++) {
	}
	printf("Succeed to count until %d\n", s);
	return 0;
}
\end{lstlisting}}
\defverbatim[colored]\myshell{%
\begin{lstlisting}[style=customshell]
$ ./intoverflow 32767
Succeed to count until 32767
$ ./intoverflow 32768
^C (infinite loop)
$
\end{lstlisting}}
\begin{frame}{Dépassement d'entier}
Un dépassement d'entier (integer overflow) est une condition qui se produit lorsqu'une opération mathématique produit une valeur numérique supérieure à celle représentable dans l'espace de stockage disponible.
\mycode
\myshell
\end{frame}


%%%%%%  Dépassement de tampon  %%%%%%
\defverbatim[colored]\mycode{%
\begin{lstlisting}[style=customc]
int main (int argc, char ** argv)
{
	char *buf2 = malloc(16), *buf1 = malloc(16);
	memcpy(buf1, argv[1], strlen(argv[1])+1);
	memcpy(buf2, argv[2], strlen(argv[2])+1);
	printf("%s\n", buf1);
	return 0;
}
\end{lstlisting}}
\defverbatim[colored]\myshell{%
\begin{lstlisting}[style=customshell]
$ ./bufoverflow hello abcdefghijklmnopqrstuvwxyz
hello
$ ./bufoverflow hello abcdefghijklmnopqrstuvwxyzabcdefghijklmnopqrstuvwxyz
ghijklmnopqrstuvwxyz
$
\end{lstlisting}}
\begin{frame}{Dépassement de tampon}
Un dépassement de tampon (buffer overflow) est un bug par lequel un processus, lors de l'écriture dans un tampon, écrit à l'extérieur de l'espace alloué au tampon, écrasant ainsi des informations nécessaires au processus.
\mycode
\myshell
\end{frame}


%%%%%%  Dépassement de pile  %%%%%%
\defverbatim[colored]\mycode{%
\begin{lstlisting}[style=customc]
int main (int argc, char ** argv)
{
	int idx=0; char buf[16];
	for(idx=0; idx <= strlen(argv[1]); idx++) {
		buf[idx]=argv[1][idx];
	}
	printf("read %d/%lu bytes\n%s\n", idx, strlen(argv[1])+1, buf);
	return 0;
}
\end{lstlisting}}
\defverbatim[colored]\myshell{%
\begin{lstlisting}[style=customshell]
$ ./stackoverflow 0123456789abcdefghijklmnopq
Read 28/28 bytes : 0123456789abcdefghijklmnopq
$ ./stackoverflow 0123456789abcdefghijklmnopqr
^C (infinite loop)
$ ./stackoverflow 0123456789abcdefghijklmnopqrs
Read 116/30 bytes : 0123456789abcdefghijklmnopqrt
$
\end{lstlisting}}
\begin{frame}{Dépassement de pile}
Un dépassement de pile (stack overflow) est un bug causé par un processus qui, lors de l'écriture dans une pile, va écraser les données précédemment pousser sur la stack.
\mycode
\myshell
\end{frame}


%%%%%%  Fuite mémoire  %%%%%%
\defverbatim[colored]\mycode{%
\begin{lstlisting}[style=customc]
int main (int argc, char ** argv)
{
	int size = atoi(argv[1]);
	while (1) {
		char * ptr = malloc(size);
		if (!ptr) // quit
			break;
		else // fill the buffer
			for(idx=0; idx < size ; idx++)
				ptr[idx]=idx;
	}
	return 0;
}
\end{lstlisting}}
\defverbatim[colored]\myshell{%
\begin{lstlisting}[style=customshell]
$ ./leak 12000000 &
$ watch cat /proc/meminfo 
...
MemFree:	xxxxxx kB # decrease until the system exit or start to swap
...
$
\end{lstlisting}}
\begin{frame}{Fuite mémoire}
Une fuite de mémoire est une occupation croissante et non contrôlée ou non désirée de la mémoire d'un ordinateur.
\mycode
\myshell
\end{frame}

%%%%%%  Situation de compétition  %%%%%%
\begin{frame}{Situation de compétition}
	\begin{itemize}
		\item
			Lorsque l'ordre d'utilisation d'une ressource partagée joue sur le résultat
		\item
			Par ex:
			\begin{itemize}
				\item
					Une variable est lue par un acteur avant d'être pourvue par un autre.
				\item
					Une fonction d'un système est exécuté avant que le système soit initialisé
			\end{itemize}
	\end{itemize}
\end{frame}

%%%%%%  Interblocage  %%%%%%
\begin{frame}{Interblocage}
	\begin{itemize}
		\item
			Lorsque un acteur détient un verrou et attend un signal d'un autre acteur attendant ce même verrou
		\item
			Plusieurs causes:
			\begin{itemize}
			\item
				Oubli de libérer le verrou
			\item
				Situation de compétition impliquant un verrou
			\item
				Mauvaise gestion des priorités entre threads
		\end{itemize}
	\end{itemize}
\end{frame}
