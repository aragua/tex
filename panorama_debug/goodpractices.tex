%%%%%%%%%%%%%%%%%%%%%%%    GOOD PRACTICES    %%%%%%%%%%%%%%%%%%%%%%%
\section{Bonnes pratiques}

\subsection{Eviter les bugs}
\begin{frame}{Éviter les bugs}
	\begin{itemize}
	 	\item
			Réfléchir avant d'agir
		\item
			Bien se documenter sur les fonctions, APIs utilisées
		\item
			Écrire du code simple, clair et modulaire
	 	\item
			Procéder étape par étape (ne pas tout coder d'un coup)			
	 	\item
			Vérifier les paramètres d'entrées dès que nécessaire
	 	\item
			Bien vérifier alloc, free, lock, unlock, ...
	 	\item
			Respecter les guidelines (Linux)
	\end{itemize}
\end{frame}

\subsection{Faciliter l'analyse}
\begin{frame}{Faciliter l'analyse}
	\begin{itemize}
	 	\item
			Utilisation de printf, zlog, syslog, printk, ftrace ...
		\item
			Utilisation de /proc ou d'un outil de stats en userland
		\item
			Wrapper des fonctions systèmes sensibles alloc, mutex, ... (facilite l'ajout de trace)
		\item
			Utiliser les mécanismes existants (stack-protector, timeout lock, RCU debug, ...)
	\end{itemize}
\end{frame}
