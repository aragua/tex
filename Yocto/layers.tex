%%%%%%%%%%%%%%%%%%%%%%%    Layers    %%%%%%%%%%%%%%%%%%%%%%%
\section{Layers}

\subsection{Layer}
\begin{frame}{Qu'est ce qu'un layer?}{}
	Ensemble de fichier contenant des informations pour construire:
	\begin{itemize}
		\item
			Logiciel
		\item
			Toolchain
		\item
			Rootfs
		\item
			...
	\end{itemize}
	En gros tout les composants d'une distribution Linux.\\
	Dans Yocto, nommé meta-<nom> par convention.\\
	Chacun peux créer son propre layer ou meta.\\
\end{frame}

\subsection{Recette}
\begin{frame}{Recette}{}
	Fichier .bb ou .bbappend\\
	Il s'agit de la description de la construction d'une entité (toolchain, logiciel, image ...)\\
	Toutes les variables ou fonctions décrites dan les recettes sont valable pour la construction de la recette.\\
	bbappend permet d'etendre une recette. (D'un autre meta par exemple)\\
\end{frame}

\subsection{Classe}
\begin{frame}{Classe}{}
	Fichier .bbclass\\
	Définit un comportement générique qu'une recette peut adopter (inherit)\\
	Toutes les variables ou fonctions décrites dan les recettes sont valable pour la construction de la recette.\\
\end{frame}

\subsection{Config}
\begin{frame}{Config}{}
	Fichier .conf\\
	Décrit des variables d'environnement globales au projet utilisable dans des recettes ou classes.\\
	Permet de decrire:
	\begin{itemize}
		\item
			la configuration local du poste de construction
		\item
			La liste des letas à utiliser
		\item
			Des machines: décris les caracteristiques du hardware
		\item
			Des distros: décris les lignes directrices de la distribution
	\end{itemize}
\end{frame}


\subsection{Inclusion}
\begin{frame}{Inclusion}{}
	Permet d'inclure à une recette des données communes par les mots-clé require ou include\\
	On peut inclure des .inc, .conf, .bb\\
	Exemples:
	\begin{itemize}
		\item
			require linux.inc
		\item
			include \path{../common/firmware.inc}
		\item
			include \path{conf/distro/include/security_flags.conf}
	\end{itemize}
\end{frame}