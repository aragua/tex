%%%%%%%%%%%%%%%%%%%%%%%    bitbake    %%%%%%%%%%%%%%%%%%%%%%%
\section{En pratique}

\defverbatim[colored]\gitclone{%
\begin{lstlisting}[style=customshell]
$ git clone git://git.yoctoproject.org/poky
\end{lstlisting}}

\defverbatim[colored]\yoctoinit{%
\begin{lstlisting}[style=customshell]
$ cd poky
$ source ./oe-init-build-env qemu_x86-build
\end{lstlisting}}

\defverbatim[colored]\yoctobuild{%
\begin{lstlisting}[style=customshell]
$ bitbake core-image-minimal
\end{lstlisting}}

\defverbatim[colored]\yoctotest{%
\begin{lstlisting}[style=customshell]
$ runqemu qemux86
\end{lstlisting}}
	
\begin{frame}{Exemple}{}
	Installation de Poky (Yocto reference distro)
	\gitclone
	Création du répertoire de travail
	\yoctoinit
	Définition (optionnelle) du type de machine dans conf/local.conf par la variable MACHINE\\
	Tous ce que bitbake construit sera dans le dossier de build.\\
	Construction d'une image simple.
	\yoctobuild
	Test de l'image
	\yoctotest
\end{frame}

\begin{frame}{Produit}{}
	Les images produites se retrouvent dans build/tmp/deploy/\\
	On y retrouve:
	\begin{itemize}
		\item
			les images dans images/ .ext3, .sdcard, .tar
		\item
			les paquets ipk, rpm, deb
		\item
			licences
		\item
			SDK
	\end{itemize}
\end{frame}

\begin{frame}{Internal}{}
	Chaque recette donne lieu a un ou plusieurs paquets.\\
	Certains pour l'hote.\\
	D'autres pour la target.\\
	Sauf les images qui elles utilisent ces paquets pour peuplés sysroot et rootfs\\
\end{frame}