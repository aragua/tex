%%%%%%%%%%%%%%%%%%%%%%%    Concept    %%%%%%%%%%%%%%%%%%%%%%%
\section{Concepts}

\begin{frame}{Concepts}{Basé sur UNIX}
	Implémentation libre d'UNIX diffusé sous licence GPL\\
	Inspiré des deux versions: AT\&T et BSD\\
	Les principes d'UNIX sont respectés:
	\begin{itemize}
		\item
			Simplicité, modularité, respect des standards, ouverture
		\item
			Deux espaces de mémoire: noyau / utilisateur
		\item
			Le noyau permet d'accéder au matériel (pilotes, appels système)
		\item
			Tout composant est un fichier: répertoire, périphérique, élément de communication, etc. (organisation arborescente)
		\item
			Puissance de la «ligne de commande» (shell et regexpr)			
	\end{itemize}	
\end{frame}

\begin{frame}{Concepts}{Fonctionnalitées}
	\begin{itemize}
		\item
			Noyau monolithique (en un seul fichier) + modules dynamiques
		\item
			Création des processus via fork() et exec()
		\item
			Multi-threading
		\item
			Nombreuses piles réseau (IPv4, IPv6, Ethernet, etc.)
		\item
			Organisation des fichiers arborescente à partir de la racine (/), montage et démontage logique (mount)
		\item
			Notion de super-utilisateur (root), groupes, et utilisateurs
	\end{itemize}
\end{frame}