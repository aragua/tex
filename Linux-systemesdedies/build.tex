\section{Construire son système embarqué}

\subsection{Bootloader}
\begin{frame}{Bootloader}{}
	Premier logiciel lancé au démarrage de la machine.
	Initialise le matériel (RAM, stockage, ...) nécésaire au boot.
	Charge le noyau en RAM et lui donne la main.
	Permet de donner des arguments au noyau.
	Exemples:
	\begin{itemize}
		\item
			U-boot (supporte de nombreux architectures, la référence)
		\item
			barebox (u-boot v2, meilleure archi, moins de fonctionnalités pour l'instant)
		\item
			grub (x86, serveur et desktop principalement)
		\item
			isolinux (alternative x86 à grub)
	\end{itemize}
	Un système embarqué nécéssitera une configuration et un support matériel dans le bootloader.
\end{frame}

\subsection{Kernel}
\begin{frame}{Kernel}{}
	Génération d'un noyau:
	\begin{enumerate}
		\item
			Récupération du noyau (kernel.org)
		\item
			Développement du support des drivers manquants.
		\item
			Création du support de la carte (board config  3.10  device tree)
		\item
			Configuration le noyau (make menuconfig)
		\item
			Compilation du noyau (make uImage)
		\item
			Installation sur la cible
	\end{enumerate}
	Binaire utilisable dans arch/arm/boot/zImage ou bien arch/arm/boot/uImage (pour U-Boot)\\
	Binaire vmlinux utile pour le debug\\
	Utilisation des variables ARCH et CROSS\_COMPILE.
\end{frame}

\subsection{Rootfs}
\begin{frame}{Binaire de base}{}
	Binaire de base indispensable au système (ls, bash, dd, top, cp, mv, init,...)
	Trois possibilités:
	\begin{itemize}
		\item
			Récupérer les sources et les compiler un par un. (laborieux)
		\item
			GNU/Linux coreutils (trop lourd pour l'embarqué)
		\item
			Busybox (regroupe la majorité des commandes Linux en un seul executable)
			95\% des distributions Linux embarqués l'utilise\\
			Simple, léger, portable\\
			Diffusé sous licence GPLv2
	\end{itemize}
\end{frame}

\begin{frame}{Busybox}{}
	Utilisation des variables ARCH et CROSS\_COMPILE. (comme le noyau)\\
	Génération de busybox:
	\begin{enumerate}
		\item
			Récupération des sources
		\item
			Configuration (make menuconfig)
		\item
			Compilation (make uImage)
		\item
			Installation sur la cible (make CONFIG\_PREFIX=... install)
	\end{enumerate}
\end{frame}

\begin{frame}{Rootfs}{Peuplement}
	Peuplement d'un rootfs:
	\begin{enumerate}
		\item
			Installation de la libc et des autres librairies fournies par la toolchain\\
		\item
			Installation des binaires de busybox
		\item
			Création de la configuration de démarrage (/etc/init.d)
		\item
			Ajout des nodes dans /dev (si pas de mécanisme automatique)
		\item
			Ajout des librairies et applications du projet.
	\end{enumerate}
	Deux mèthodes:
	\begin{itemize}
		\item
			A la main, copier ou installer chaque fichier/logiciel manuellement. (LFS)\\
			Inenvisageable à moyen et long terme
		\item
			Automatiser soit en scriptant soit en utilisant un système existant (OE, yocto, buildroot, uclinux ...)\\
			Plus rapide et plus sure.\\
			Indispensable en production pour maitriser son livrable.
	\end{itemize}
\end{frame}


