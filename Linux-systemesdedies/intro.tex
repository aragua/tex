%%%%%%%%%%%%%%%%%%%%%%%    Intro    %%%%%%%%%%%%%%%%%%%%%%%
\section{Introduction}

\subsection{Systèmes dédiés?}
\begin{frame}{Intro}{Systèmes dédiés?}
	Un système dédié est un système concu pour ne réaliser qu'un certains nombre de tache défini.\\
	Exemple:
	\begin{itemize}
		\item
			Serveur web
		\item
			Calculateur automobile
		\item
			Box multimedia
	\end{itemize}
	On associe un système à une fonction (ou un groupe de fonction).\\
	On cherche à maitriser le fonctionnement du système.\\
	Différe des systèmes génériques censés savoir tout faire sur n'importe quelle plateforme.\\
\end{frame}

\begin{frame}{Intro}{Systèmes dédiés GNU/Linux?}
	Utilisé depuis longtemps comme serveur.\\
	Pourquoi ne pas l'utiliser pour d'autre système générique? (systèmes embarqués)\\
	De nombreuses architectures supportées par Linux (ARM, mips, powerpc ...)\\
	Systèmes fiables et performants\\
	De nombreuses fonctionnalités déjà codé et validé\\
	Nécessite un BSP (Board support package) et un peu d'adaptation.\\
\end{frame}

\subsection{Eléments nécéssaires}
\begin{frame}{Intro}{Eléments nécéssaires}
	Une système GNU/Linux dédié nécéssite:
	\begin{itemize}
		\item
			Chaîne de compilation croisée (Gcc,as,ld,LibC)
		\item
			Un Bootloader (U-Boot, barebox, grub, isolinux)
		\item
			Noyau Linux adapté à l'archi hardware
		\item
			Outils GNU/LINUX Commandes Linux (sh, ls, cp, etc.)\\
			Applications ...
		\item
			Outil de génération (BR, OE, ...)
	\end{itemize}
\end{frame}