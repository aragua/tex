\section{Mise au point}

\subsection{BSP(uboot+linux)}
\begin{frame}{BSP}{}
	Une plateforme "neuve" nécéssite des ajustements pou fonctionner:
	\begin{itemize}
		\item
			Adaptation de la toolchain
		\item
			Support des drivers dans le bootloader et le kernel
		\item
			Ajout des fonctionnalités dans le bootloader et le kernel (réseau, NAND, HDD, ...)
	\end{itemize}
	Fourni par le constructeur de la plateforme éléctronique.
	Si on est le constructeur, il faut développer le support.
	Outils à disposition (gdb, sonde JTAG, ftrace, oscilloscope, multimetre ...) 
\end{frame}
			
\subsection{User space}
\begin{frame}{User space}{}
	Lorsque le BSP est OK, la mise au point en espace utilisateur peut commencer.\\
	De nombreux outils permettent de mettre au point des applications:
	\begin{itemize}
		\item
			printf: outil simple à mettre en oeuvre et connu de tout le monde
		\item
			syslog: utile pour vérifier le bon fonctionnement ou detecter des anomalies
		\item
			valgrind: vérifie la bonne utilisation de la mémoire
		\item
			strace/ltrace: affiche les appels systèmes et librairies
		\item
			gdb: debugguer GNU/LINUX\\
			permet d'acceder à la memoire, controler l'execution, monitoré les threads ...\\
			gdbserver pour les cibles embarqués pour déporter le debug
	\end{itemize}
\end{frame}
