\section{Toolchain}

\begin{frame}{Toolchain}{}
	Un point très complexe !\\
	Nécessité de construire une chaîne croisée :\\
	\begin{itemize}
		\item
			Gcc
		\item
			Binutils (as, ld, ...)
		\item
			Dépendances avec le noyau (system calls, ...) => erreur "Kernel too old"
		\item
			Choix d'une libC => Glibc, uClibc, Eglibc, ...
		\item
			GDB
		\item
			Toute autre bibliothèque utilisée => libstdc++
		\item
			Dépendances avec le compilateur hôte
	\end{itemize}		
\end{frame}

\begin{frame}{Toolchain}{}
	Interaction entre la libC et le noyau Linux
	\begin{itemize}
		\item
			Appels systèmes (nombre, définition)
		\item
			Constantes 
		\item
			Structures de données, etc.
	\end{itemize}
	Compiler la libC – et certaines applications - nécessite les en-tête du noyau\\
	Disponibles dans <linux/...> et <asm/...> et d'autres répertoires des sources du noyau (include, ...)\\
\end{frame}

\subsection{Compilateur binaire}
\begin{frame}{Toolchain}{compilateur binaire}
	Utiliser un compilateur binaire :
	\begin{itemize}
		\item
			ELDK: http://www.denx.de/wiki/DULG/ELDK
		\item
			Code Sourcery : https://sourcery.mentor.com/sgpp/lite/arm/portal/release1803
		\item
			Installation simple
		\item
			Support (payant) possible
		\item
			Configuration connue => support par les forums
	\end{itemize}
	Par contre:
	\begin{itemize}
		\item
			Versions des composants figées
		\item
			Non utilisation des possibilitées du CPU
		\item
			Choix libC limité
	\end{itemize}
\end{frame}

\subsection{Compilateur source}
\begin{frame}{Toolchain}{Compilé son compilateur}
	Construire un compilateur:
	\begin{itemize}
		\item
			Crosstool => obsolète
		\item
			Crosstool-NG => assez complexe à prendre en main
		\item
			Buildroot / OpenEmbedded
	\end{itemize}
	Aucun n'est "plug and play"\\
	La mise au point peut prendre des jours, voire plus !\\
	Binaires produits : arm-linux-* (gcc, as, ld, ar, nm, ...)\\
\end{frame}

