\documentclass[11pt,a4paper,sans]{moderncv}

%% ModernCV themes
%\moderncvstyle{casual}
\moderncvstyle{classic}
\moderncvcolor{red}
\renewcommand{\familydefault}{\sfdefault}
\nopagenumbers{}

%% Character encoding
\usepackage[utf8]{inputenc}

%% Adjust the page margins
\usepackage[scale=0.75]{geometry}

\usepackage{url}

%% Personal data
\firstname{Fabien}
\familyname{Lahoudere}
\title{Développeur Linux embarqué}
\email{fabienlahoudere.pro@gmail.com}
\address{26 rue de Cruchen}{31490 Léguevin}
\mobile{+336.60.17.09.53}
%\phone{}
%\fax{}
%\homepage{aragua.ddns.net}
%\extrainfo{additional information}
%\photo[64pt][0.4pt]{toto}
%\quote{Some quote (optional)}

%%------------------------------------------------------------------------------
%% Content
%%------------------------------------------------------------------------------
\begin{document}
\makecvtitle

\section{Expérience formateur Linux embarqué}
\cventry{Nov. 2014}{Formateur Linux}{INSA}{Toulouse}{}{Initiation à Linux embarqué, OpenEmbedded}
\cventry{Mai 2014}{Formateur Linux}{Twiga}{Toulouse}{}{Introduction à Linux embarqué, Yocto, compilation croisée dans QT\\
Mise en pratique par le développement d'une distribution contenant une application de capture/affichage vidéo}
\cventry{Nov. 2013}{Formateur Linux}{INSA}{Toulouse}{}{Initiation à Linux embarqué, OpenEmbedded}

\section{Expérience développeur Linux}
\cventry{Juin 2014\\Aujourd'hui}{Développeur Linux}{Actia (Open Wide)}{Toulouse}{}{Spécification et développement en C d'une communication inter micro basée sur SPI\\
Développement d'un netdevice CAN/SPI et adaptation API socket CAN\\
Mise en place et support du build system (Yocto)}

\cventry{Fév. 2014\\Mai 2014}{Développeur Linux et réseau}{Arkoon \& Open Wide}{Toulouse}{}{Intégration de netmap dans HAKA security\\
\url{https://github.com/haka-security/haka}}

\cventry{Janv. 2014\\Fév. 2014}{Intégrateur Yocto}{Nexter systems (Open Wide)}{Toulouse}{}{Création d'une distribution embarquée avec QT sur x86\\
Intégration de l'application client}

\cventry{Oct. 2014\\Janv. 2014}{Développeur Linux embarqué}{ECA Sinters (Open Wide)}{Toulouse}{}{Mise en place du BSP et du build system sur carte d'évaluation\\
Développement en C d'une communication inter micro sur ethernet\\
Développement d'un driver de communication FPGA<->Cortexm3\\
Démarrage de la carte finale et debug des drivers}

\cventry{Fév. 2013\\Sept. 2013}{Développeur Linux embarqué}{ESII (Open Wide)}{Toulouse}{}{Développement en C d'un streamer audio sur wifi\\
Intégration sur carte à base imx53 (OpenEmbedded)\\
Développement d'une application cliente sur Android (NDK)}

\cventry{Juin 2012\\Janv. 2013}{Développeur et intégrateur Linux}{M3 Systems (Open Wide)}{Toulouse}{}{Développement en C d'un outil de collecte régulière de données GNSS\\
Intégration sur PC (debootstrap puis buildroot) et portage sur système embarqué }

\cventry{Janv. 2012\\Mai 2012}{Développeur Linux}{Zodiac aerospace (Open Wide)}{Toulouse}{}{Etude et développement d'un système de VOD prépayé pour IFE\\
Développement en C, d'un plugin de déchiffrement pour vlc, mise en place d'une PKI et d'un outil de chiffrement et développement d'une solution de prépaiement}

\cventry{Juil. 2010\\Déc. 2011}{Développeur Linux embarqué}{Sagemcom (Open Wide)}{Rueil-Malmaison}{}{Développement de drivers d'interface avec les composants de sécurité d'une ``set top box'' (PVR, descrambling, access control, chiffrement NAND)\\
Développement d'un outil de SAV passant des commandes chiffrées à la STB après authentification mutuelle (SSL)\\
Intégration de drivers pour la compatibilité avec de nouvelles NAND gérant l'ECC en interne\\
Récupération de box sécurisées sans JTAG flashées avec des briques corrompues}

\cventry{Nov. 2009\\ Juin 2010}{Consultant sécurité}{Amesys RSS (Key Consulting)}{Boulogne-Billancourt}{}{Sécurisation d'une application (C/perl) sous RedHat contre le reverse engineering}

\cventry{Oct. 2008\\Oct. 2009}{Développeur Linux}{Intego}{Paris}{}{Création et test de la partie bas niveau d’un antivirus en C sous Linux\\
Portage et test de la librairie antivirale\\
Spécification et implémentation d’un scanner de fichier\\
Interception en temps réel des accès au système de fichiers (module noyaux)}

\cventry{Avril 2008\\Sept. 2008}{Stagiaire}{Thomson R\&D}{Rennes}{}{Etat de l’art des techniques de protection contre le désassemblage\\
Etude de leurs contre-mesures basées sur des heuristiques\\
Réflexion sur de nouvelles techniques\\
Création d’un outil en C++ pour automatiser l’application de ces méthodes}

\section{Formation}
\cventry{2008}{Master}{Université Bordeaux1}{Talence}{}{Cryptologie et sécurité informatique}
\cventry{2006}{Licence}{Université Bordeaux1}{Talence}{}{Mathématiques et informatique} 
\cventry{2003}{Baccalauréat}{Lycée Gustave Eiffel}{Bordeaux}{}{Scientifique sciences de l'ingénieur, spécialité Mathématiques}  

\section{Langues}
\cvitemwithcomment{Français}{Langue maternelle}{}
\cvitemwithcomment{Anglais}{Opérationnel}{}
\cvitemwithcomment{Espagnol}{Débutant}{}

%\cvdoubleitem{category X}{XXX, YYY, ZZZ}{category Y}{XXX, YYY, ZZZ}
%\cvlistitem{Item 1}
%\cvlistdoubleitem{Item 2}{Item 3}
%% ...

%\bibliography{publications}
\end{document}